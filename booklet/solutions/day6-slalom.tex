\bigheading{An inexperienced slalomer}

\authors{Peter `Bob' Fulla}{Peter `Bob' Fulla, Marián `Syseľ' Horňák}{Peter `Bob' Fulla}

A gate has two endpoints; we will call the one with smaller $y$-coordinate the
\emph{bottom} endpoint and the other one the \emph{top} endpoint. In order to
pass through each gate, Hubert must ski above every bottom endpoint and below
every top endpoint. He moves along a straight line, therefore we may replace
this requirement by ``Hubert must ski above the upper convex hull of bottom
endpoints and below the lower convex hull of top endpoints'', which is
equivalent to it.

If the intersection of the two convex hulls has a positive area, the answer will
be \texttt{Impossible}, as no straight line will separate the hulls. Otherwise,
let us choose a point $P$ belonging to one of the hulls and a point $Q$
belonging to the other hull such that their distance is minimal among all such
pairs of points; we will denote their distance by $d$. In other words, $d$
equals the distance between the two convex hulls. We claim that $d$ is also the
answer to our problem. Clearly, a disk of a diameter strictly greater than $d$
cannot pass through the line segment $PQ$, therefore its trajectory cannot
separate the two convex hulls. On the other hand, there is a valid trajectory
for a disk of diameter exactly $d$: Its center will move along the perpendicular
bisector of the line segment $PQ$. One can show that this trajectory does not
intersect the two convex hulls (though it touches them) -- if it did, there
would be a pair of points $P'$, $Q'$ with a smaller distance than $d$.

The intersection of the two convex hulls has a positive area iff a vertex of one
hull lies strictly inside the other hull. We can check this by a simple sweep
line algorithm. The distance of the convex hulls can be computed as follows:
Without loss of generality, we may assume that at least one of the points $P$,
$Q$ is a vertex of the convex hull it belongs to; the other may also be a vertex
or it may lie on a line segment on the hull's border. For any fixed vertex $V$
on a convex hull, we can compute its distance to the other hull $H$ in time
$O(n)$ simply by iterating through all vertices and line segments of $H$ and
taking the minimum of obtained distances. This would lead to a solution with
time complexity $O(n^2)$. Realizing that the sequence of distances from a fixed
vertex $V$ is at first decreasing and then increasing, we can find its minimum
using a ternary search in time $O(\log n)$. Another option is to use the two
pointers technique to find minima for all vertices $V$ in time $O(n)$. As we
have to sort the gates at the beginning, the overall time complexity is in both
cases $O(n\log n)$.
