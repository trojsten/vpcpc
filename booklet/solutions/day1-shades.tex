\bigheading{Shades of the town}

\authors{FIXME Problem Setter}{FIXME Testdata Developers}{FIXME Writeup Author}

We will first discuss a simple brute-force solution.
Then, we will show how to reduce this problem to a standard string matching problem,
and then we will solve that problem.

\heading{Solution using brute force}

Given a pattern of length $\ell_i$, we can simply try all $m-\ell_i+1$ possible offsets
between the pattern and the sequence of shadows. For a fixed offset, we simply verify
whether all pairs (building,shadow) give the same scaling factor. This can be done
in $O(\ell_i)$.

Thus, the total time complexity of the brute force solution is $O(m\cdot\sum\ell_i)$.

\heading{Reduction to string matching}

The main trick in solving this task is getting rid of the scaling. To do that, we can
make the following observation: the only thing that matters are ratios of consecutive
building heights / shadow lengths.

\bigskip

Here's the same thing formally.

Notation: Given a sequence $A=(a_1,\dots,a_k)$ of positive reals, the sequence $\rho(A)$ 
(read ``ratios of $A$'') is defined as follows: $(a_2/a_1,\, a_3/a_2,\, \dots,\, a_n/a_{n-1})$.

Lemma: Given two sequences $A=(a_1,\dots,a_k)$ and $B=(b_1,\dots,b_k)$ of positive reals,
the sequence $B$ can be obtained by scaling $A$ if and only if $\rho(A)=\rho(B)$.

Proof of the lemma should be obvious.

\bigskip

Thus, instead of solving the original problem, we can compute the ratios of all patterns
and look for those in the ratios of the shadow lengths. Thus we get a standard string
matching problem: given a long string (``haystack'') and a collection of short strings
(``needles''), count the number of occurrences of all needles in the haystack.
Of course, the \emph{alphabet} we are using are actually \emph{fractions}: each 
``letter'' of our strings is actually a rational number.

This standard string matching problem can be solved optimally using the Aho-Corasick algorithm
The time complexity of this solution is $O(m + \sum\ell_i)$, that is, linear in the input size.

A slightly slower but slightly simpler solution is to use an $O(m+\ell_i)$ string matching 
algorithm (such as KMP or Rabin-Karp hashing) separately for each pattern. With $n$ patterns,
this gives us the total time complexity $O(nm + \sum\ell_i)$.

\heading{The Aho-Corasick algorithm}

The Aho-Corasick algorithm can be seen as a generalization of the Knuth-Morris-Pratt algorithm
for a single pattern. In Aho-Corasick we insert all patterns into a trie, and then build a set
of back-links that corresponds to the failure function of the KMP algorithm. Formally, for each
node $v$ in the trie (other than the root) we have a single back-link that points to the deepest
node $w$ that is less deep than $v$, and has the property that the string that corresponds to $w$
is a suffix of the string that corresponds to $v$. (I.e., if, after processing some letters 
of the haystack, we can be in the node $v$, we can also be in the node $w$.)

Some care must be taken when counting the matches. For example, if we are looking for the needles
\verb!abcd! and \verb!bc! in the haystack \verb!pqrabcuvw!, after processing the letters
\verb!pqrabc! we will be in the trie node that corresponds to the string \verb!abc! -- however,
at this moment we should count one occurrence of \verb!bc!. In a full implementation of Aho-Corasick
this is handled using a second type of back-links (``output links''), but in our case we can simply
precompute this information in linear time once we have the trie with back-links ready.

