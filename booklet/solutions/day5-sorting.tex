\bigheading{Sorting}

\authors{JAG Practice Contest, Tokyo, 2011-11-06}{Marek Sommer \& Błażej Magnowski}{Marek Sommer}

Let's see what would happen if we would put the numbers $A$, $A+1$, $\ldots$, $B-1$, $B$ into an array,
	and then sort them lexicographically
		(sorting may be done just by converting each number into a string, and then comparing two strings letter by letter).
We may call this sorted sequence, a sequence $S$.
Let $n$ be the length of the sequence $S$ (and also the size of set $\{A, A+1, \ldots, B-1, B\}$).

If we would look at any ascending subsequence of $S$, and if we would consider
	a set containing numbers from this subsequence, then this set would meet conditions stated in the task.
That's because this subsequence is in lexicographical order (as a subsequence of a sequence with lexicographical order),
	and also it is in numerical order (because it's ascending).

From the other side, if we look at any subset of $\{A, A+1, \ldots, B-1, B\}$,
	with the given property, and we would sort it lexicographically,
	then it would form a subsequence of sequence $S$
		(because $S$ contains all elements of this subsequence and these elements are correctly ordered).
Moreover, this sequence would be ascending, because the set has the property, that lexicographical and numerical
	orders are the same.
Hence, this subset would be formed from an ascending subsequence of sequence $S$.

So, the number of subsets of $\{A, A+1, \ldots, B-1, B\}$ with the given property
	is equal to the number of ascending subsequences of sequence $S$.

To find the number of such subsequences we will use an algorithm very similar to the algorithm
	which computes the longest ascending subsequence.
We will use dynamic programming and for each element of the sequence $S$,
	we will compute the number of ascending subsequences, ending exactly on this element.
Let's call this value $p_i$ for $i$-th element.
If we would be able to compute such values, then the answer would be the sum of them
	(as each ascending subsequence should end in one of those elements), plus one (we must add empty subsequence).

We will compute these values starting from the first element and ending on the last one.
The value $p_1$ is always $1$ (no matter what the first element is).
If we try to compute the answer for the $i$-th element ($S_i$),
	then we must find the sum of all $p_j$ such that $j < i$ and $S_j < S_i$.
This sum is the number of all ascending subsequences, ending in $i$-th element, of length greater than one.
Therefore we must increase this sum by one to obtain the number of all subsequences ending in the $i$-th element.

At this point we already have an $O(n^2)$ solution.
We can make it better by finding the sums of $p_j$ more efficiently then by checking each possible $j$.

Let's create an array of length $n$.
First element of this array would represent $A$, second element would represent $A+1$, and so on.
After we compute a value $p_i$, we should store it in the array in the element representing $S_i$.
If we would like to compute the sum of $p_j$ with $j < i$ and $S_j < S_i$,
	then we should simply calculate the sum of array's elements from the one representing $A$ to the one representing $S_i - 1$.
This continuous part of the array contains $p_j$ values such that $S_j < S_i$.
The second constraint that $j < i$ is simply hold thanks to the order in which we compute the values $p_i$
	(from $i=1$, to $i=n$), so in the array there are only $p_j$ such that $j < i$.

We reduced the problem to construction of an array, on which we could make two queries:
	setting one element, and computing a sum of elements on a range of elements.
This could be done with a segment tree, or a Fenwick tree, giving us an $O(n\log n)$ algorithm
	to count the number of ascending subsequences.
The overall solution is a bit slower, because the lexicographical sorting was done in $O(n\log n \log B)$ time.
