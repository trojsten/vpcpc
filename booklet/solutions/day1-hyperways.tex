\bigheading{Hyperways}

\authors{Marián Horňák}{Marián Horňák}{Michal Forišek, Marián Horňák}

Formally, planets are graph vertices, hyperways are edges and unsafe hyperways
are bridges in the graph. Graph is not directed but can contain multi-edges and
self-loops. Our task is to output how many edges are not more bridges after
each edge addition.

\heading{Slow solutions}

Basic idea of slow solutions is to recalculate number of bridges after each
edge addition. You can do it in quadratic time deleting each edge and running
BFS / DFS to find out whether it has changed the number of components.
(The edge is bridge if and only if you split the graph deleting it.)
This solution runs in $O(m^3)$ time.

To improve it, you can find all the bridges running a single DFS. You need to
remember the ``time'' you first visited each vertex. It is called visit time.
Time can be for example the number of vertices you visited before.
When returning from
vertex $v$ back to vertex $u$ using some edge $e$,
consider the earliest visit time you have met since you entered $e$.
Lets call it return time. Return time can be calculated
as the minimum of visit time of $v$ and the return times
of all the other edges incident to $v$. If this wasn't first visit of
$v$, we returned directly and it is just $v$'s visit time.

If the return time of $e$ is higher than (or equal to) the first visit time
of $u$, vertices $u$ and $v$ can be connected through the vertex this
time belongs to and thus $e$ is not a bridge.
Otherwise, part of graph we visited has no other connection to the rest
and $e$ is a bridge. Therefore, we can easily decide whether $e$ is a bridge
or not. Solution works in $O(m^2)$ time.

\heading{Union Find algorithm}

Union Find algorithm is a prerequisite to the optimal solution
Since it is well known algorithm, I just mention it briefly.

Imagine you have a number of disjoint sets of elements.
Union Find allows you to perform two operation on these sets:
join two sets into one and find out the set given element is in.
It works in amortized time complexity $O(k\cdot \alpha(k))$,
where $k$ is the number of operations and $\alpha$ is the inverse
of extremely fast-growing Ackermann function.

\heading{Optimal solution}

Union Find can be used to determine the component vertex belongs to.
When adding an edge that connects two components, this edge must
be a bridge and thus the number of edges that are no more bridges is $0$.

Problem is with edges that connects vertices from the same component.
Suppose, the first such edge was added. Since, it is the first one,
graph was a forest before and this edge creates a cycle $c$. All the edges
in the cycle are no more bridges, so we can output cycle size.

Imagine we have added few new edges and there is at least one edge $e$
incident to some vertex of the cycle $c$. Now we move $e$ to the another
vertex of the cycle $c$. This operation will not create or remove any
new cycles, because all the cycles that may be affected intersect with $c$
before as well as after the operation. Therefore, this operation does not
bridgeness of any edge. This allows us to do not care
about which cycle we are using and thus merge cycle $c$ into one vertex.

Since $c$ is the only cycle in the time of its creation,
merge will make our graph the forest again.
Repeating this, we can process all the vertices.

The only question is, how to merge the cycle.

\heading{Cycle merging}

Union Find can be used to merge the vertices. The problem is to
calculate the cycle.

Consider the spanning forest of the complete graph, created by
by skipping all the edges, that would create a cycle. This can be easily
precalculated (using Union Find to check whether next edge will be bridge).
Then one can select root in every tree of the forest and use DFS
to direct the edges to the root.

When the first non-bridge edge comes, all edges in the real graph
are bridges. Therefore, they are subset of the edges in the precalculated
directed spanning forest and the created cycle is the same in both graphs.
In the directed graph we are able to find the lowest common ancestor (LCA) of
connected vertices which is the top vertex of the cycle.

LCA can be found by repeatedly moving deeper vertex one generation up
(depths can be also precalculated).

After the cycle is found, vertices are merged also in precalculated graph.
This does not affect the the structure of the graph and we can suppose
they were never split so the above argumentation can be used again.

Every merge decreases the number of mergeable elements by one.
There are three levels of merges: precalculation, components and vertex
merges. The number of Union find requests is linear to the number of edges
and the number of merges. DFS runs in linear time. Finding LCA visits
each vertex once (then is merged). Therefore, the time complexity
is $O(m\cdot \alpha(m))$.
