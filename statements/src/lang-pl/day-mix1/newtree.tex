\input sys/inputs.tex

\usepackage[utf8]{inputenc}
\usepackage[T1]{fontenc}
\usepackage[polish]{babel}
\usepackage{polski}

\begin{document}

\bigheading{New Tree}

% \info{task_name}{infile}{outfile}{points}{timelimit}{memlimit}
% leave this values, if you are not interested
\info{newtree}{stdin}{stdout}{100}{200 ms}{1 GB}

W parku, w centrum miasta, zostało posadzone nowe drzewo.
Ogrodnik pragnie się nim starannie opiekować.
Chce zbudować wokół niego ogrodzenie, które będzie je chroniło.
Zbuduje je w ten sposób, że wybierze pewne
	trzy stare drzewa\footnote{drzewa, które rosły przed posadzeniem nowego drzewa -- przyp. red.}
	i rozciągnie między nimi siatkę.
Nowe drzewo powinno wtedy znaleźć się wewnątrz ogrodzonego terenu
	i żadne inne drzewo nie ma prawa już tam rosnąć.

Ogrodnik wybrał już jedno ze starych drzew do zbudowania ogrodzenia.
Pomóż mu znaleźć dwa pozostałe.

\heading{Task}

Twoim zadaniem jest znaleźć dwa stare drzewa, które wspólnie z tym wybranym będą odgradzały teren,
	w~którym będzie się znajdować nowe drzewo i żadne inne.

\heading{Input}

W pierwszej linii wejścia znajdują się dwie liczby całkowite $N$ i $A$ ($3 \le N \le 100\,000$, $1 \le A \le N$),
	oznaczające liczbę starych drzew i numer drzewa, wybranego przez ogrodnika (drzewa są ponumerowane od $1$ do $N$).
W drugiej linii znajdują się dwie liczby $x$ i $y$ -- współrzędne nowego drzewa.
W każdej z następnych $N$ linii znajdują się dwie liczby $x$ i $y$ -- współrzędne starego drzewa.
Dla wszystkich współrzędnych zachodzi warunek: $-1\,000\,000 \le x, y \le 1\,000\,000$.

W $40\%$ testów zachodzi dodatkowy warunek $N \le 5000$.

\heading{Output}

Na wyjście należy wypisać dwie liczby całkowite $B$ i $C$, oddzielone pojedynczym odstępem.
Liczby $B$ i $C$ mają oznaczać numery starych drzew, takie, że jeśli $A$ jest numerem drzewa wybranego przez ogrodnika,
	to trójkąt o wierzchołkach w drzewach $A$, $B$ i $C$ (\textbf{w kolejności przeciwnej do wskazówek zegara})
	tworzą poprawny teren do ochrony nowego drzewa.
Nie mogą istnieć drzewa (różne od $A$, $B$ i $C$) na bokach tego trójkąta,
	oraz jedynym drzewem leżącym \textbf{ściśle} wewnątrz trójkąta musi być nowe drzewo.

Jeśli nie można wybrać takich drzew $B$ i $C$, to należy wypisać \texttt{0 0}.

Jeśli istnieje wiele poprawnych rozwiązań, należy wypisać dowolne z nich (nie ma znaczenia które).

\heading{Sample}

\sampleIN
7 1
9 3
3 1
8 7
9 5
11 5
12 4
9 1
13 6
\sampleOUT
6 4
\sampleEND

\end{document}
