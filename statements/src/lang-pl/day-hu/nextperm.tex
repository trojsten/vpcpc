\input sys/inputs.tex

\usepackage[utf8]{inputenc}
\usepackage[T1]{fontenc}
\usepackage[polish]{babel}
\usepackage{polski}

\begin{document}

\bigheading{Next Permutation}

% \info{task_name}{infile}{outfile}{points}{timelimit}{memlimit}
% leave this values, if you are not interested
\info{nextperm}{stdin}{stdout}{100}{100 ms}{32 MiB}

Permutacje są bardzo ważnym bytem w środowisku matematycznym i informatycznym,
	a permutacje, które nie zawierają pewnych wzorców są szczególnie ciekawe.
Permutacja $p_1$, $p_2$, $\ldots$, $p_n$ liczb naturalnych $1$, $\ldots$, $n$
	nazywa się permutacją bez wzorca 3-1-2, jeśli nie istnieją indeksy $i$, $j$, $k$ ($1 \le i < j < k \le n$),
	takie, że $p_i > p_j$, $p_i > p_k$ i $p_j < p_k$.

\heading{Task}

Napisz program, który dla danej permutacji bez wzorca 3-1-2 znajdzie następną
	taką permutację w kolejności leksykograficznej.

\heading{Input}

W pierwszej linii wejścia znajduje się jedna liczba $n$ ($3 \le n \le 10\,000$).
W drugiej linii znajduje się $n$ liczb całkowitych, pooddzielanych pojedynczymi spacjami.
Jest to permutacja liczb $1$, $\ldots$, $n$, bez wzorca 3-1-2.
Permutacja z wejścia nie jest permutacją $n$, $n-1$, $\ldots$, $1$.

W $40\%$ testów zachodzi $n \le 1000$.

\heading{Output}

W pierwszej linii wyjścia należy wypisać permutację bez wzorca 3-1-2, która następuje po permutacji z~wejścia.
Liczby powinny być oddzielone pojedynczym odstępem.

\heading{Sample}

\sampleIN
5
2 4 5 3 1
\sampleOUT
2 5 4 3 1
\sampleEND

\end{document}
