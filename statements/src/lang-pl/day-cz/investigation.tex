\input sys/inputs.tex

\usepackage[utf8]{inputenc}
\usepackage[T1]{fontenc}
\usepackage[polish]{babel}
\usepackage{polski}

\begin{document}

\bigheading{Investigation}

% \info{task_name}{infile}{outfile}{points}{timelimit}{memlimit}
% leave this values, if you are not interested
\info{investigation}{stdin}{stdout}{100}{2500 ms}{1 GB}

W Bajtocji, na wolności grasuje straszny złodziej.
Jesteś detektywem i Twoim zadaniem jest go znaleźć i~aresztować.

Bajtocja składa się z $N$ domów, połączonych $N-1$ drogami,
	które łączą pewne domy w taki sposób, że między dowolną parą domów istnieje dokładnie jedna ścieżka
	(czyli Bajtocja ma strukturę drzewa).
Złodziej ukrywa się w jednym z tych domów.

Aby zlokalizować złodzieja, możesz wybrać dowolny dom $h$ i go przeszukać.
Jeśli w tym domu znajdował się złodziej, to aresztujesz go.
W przeciwnym przypadku, przesłuchujesz domowników, a oni naprowadzają Cię na jego trop,
	mówiąc w którym kierunku od danego domu znajduje się złodziej.
Jeśli wyobrazisz sobie miasto jako ukorzenione drzewo, o korzeniu $h$, którego dziećmi
	są domy $c_1$, $c_2$, $\ldots$, $c_m$, to domownicy powiedzą, w którym z poddrzew $c_i$
	kryje się złodziej.

Dopóki nie znajdziesz złodzieja, będziesz przeszukiwać kolejne domy.
Możesz założyć, że w trakcie całego śledztwa złodziej nie opuści domu, w którym się ukrył.

Oczywistym jest, że kolejność przeszukiwania domów ma znaczenie,
	ponieważ nawet jeśli nie znajdziesz złodzieja w jednym domu, to informacja, którą uzyskasz od domowników
	może bardzo zawężyć podejrzany zbiór domów.
Twoim zadaniem jest znaleźć optymalną strategię, która minimalizuje liczbę przeszukanych domów
	w~najgorszym możliwym przypadku.

\heading{Task}

Masz dany opis miasta.
Znajdź optymalną strategię, która minimalizuje liczbę przeszukanych domów w~najgorszym możliwym przypadku.

\heading{Input}

W pierwszej linii wejścia znajduje się jedna liczba $N$ ($2 \le N \le 10^5$), oznaczająca liczbę domów w Bajtocji
	(domy są ponumerowane od $0$ do $N-1$).

W drugiej linii znajduje się $N-1$ liczb całkowitych $v_1$, $v_2$, $\ldots$, $v_{N-1}$.
Liczba $v_i$ ($1 \le i \le N-1$) oznacza, że istnieje droga łącząca domy o numerach $v_i$ oraz $i$ ($v_i < i$).

W $20\%$ testów zachodzi $N \le 10$.

W $40\%$ testów zachodzi $N \le 20$.

W $60\%$ testów zachodzi $N \le 1\,000$.

\heading{Output}

Na wyjście należy wypisać jedną liczbę całkowitą -- liczbę domów, które musisz odwiedzić w najgorszym możliwym przypadku,
	podczas gdy stosujesz optymalną strategię.

\heading{Samples}

\sampleIN
5
0 1 1 1
\sampleCOMMENT
Bajtocja jest gwiazdą, z domem $1$ w środku.
\sampleOUT
2
\sampleCOMMENT
Najpierw należy przeszukać dom $1$.
Jeśli złodzieja tam nie ma, to informacja uzyskana od mieszkańców, wskaże dom, w którym się ukrywa.
\sampleEND

\sampleIN
8
0 1 2 1 3 5 6
\sampleOUT
3
\sampleEND

\end{document}
