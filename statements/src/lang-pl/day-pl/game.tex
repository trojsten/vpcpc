\input sys/inputs.tex

\usepackage[utf8]{inputenc}
\usepackage[T1]{fontenc}
\usepackage[polish]{babel}
\usepackage{polski}

\begin{document}

\bigheading{Game}

% \info{task_name}{infile}{outfile}{points}{timelimit}{memlimit}
% leave this values, if you are not interested
\info{game}{stdin}{stdout}{100}{100 ms}{32 MiB}

Mirek lubi się bawić liczbami.
Razem ze swoim przyjacielem Kamilem gra w następującą grę.
Na początku są dwie liczby całkowite $A$ i $B$.
Powiedzmy, że $A \le B$.
Gracze mogą na zmianę wykonywać jeden z wybranych ruchów:
\begin{itemize}
	\item Zastąp liczbę $B$ liczbą $B - A^K$, gdzie $K$ może być dowolną liczbą całkowitą (wybraną przez gracza),
		spełniającą: $K > 0$ i $B - A^K \ge 0$.
	\item Zastąp liczbę $B$ liczbą $B \bmod A$.
\end{itemize}

Jeśli $B \le A$, podobne ruchy są możliwe.
Gracz, który zastąpi dowolną liczbę, liczbą $0$, wygrywa.
Mirek zawsze zaczyna.
Lubi grać w tę grę, ale gdyby zawsze wygrywał to dopiero wtedy zacząłby być naprawdę szczęśliwy.
Pomóż mu stwierdzić kto wygra, jeśli obaj gracze grają optymalnie.

\heading{Task}

Masz dany opis gier, w które Mirek i Kamil chcą zagrać.
Dla każdej gry stwierdź, kto wygra.

\heading{Input}

W pierwszej linii wejścia znajduje się jedna liczba całkowita $T$ ($1 \le T \le 10^4$),
	oznaczająca liczbę gier, w~które chłopaczki chcą zagrać.
W następnych $T$ liniach znajdują się opisy poszczególnych gier.
Każda z tych linii zawiera dwie liczby całkowite $A$, $B$ ($1 \le A, B \le 10^{18}$).

W $30\%$ testów $A, B \le 1000$.

\heading{Output}

Wypisz $T$ linii.
W $i$-tej z nich powinno znajdować się imię zwycięzcy $i$-tej gry: \texttt{Mirek} albo \texttt{Kamil}.

\heading{Sample}

\sampleIN
4
1 1
12 4
4 6
15 31
\sampleOUT
Mirek
Mirek
Kamil
Mirek
\sampleCOMMENT
Okazuje się, że Mirek ogra Kamila.
\sampleEND

\end{document}
