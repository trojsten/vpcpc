\input sys/inputs.tex

\usepackage[utf8]{inputenc}
\usepackage[T1]{fontenc}
\usepackage[polish]{babel}
\usepackage{polski}

\usepackage{tikz}

\begin{document}

\bigheading{Posters}

% \info{task_name}{infile}{outfile}{points}{timelimit}{memlimit}
% leave this values, if you are not interested
\info{posters}{stdin}{stdout}{100}{2000 ms}{1 GB}

Mirek jest oddanym fanem swojego ulubionego zespołu.
Jeździ na ich wszystkie koncerty i kupuje drogie plakaty z ich wizerunkiem.
Za każdym razem, gdy zdobędzie nowy plakat, wiesza go na ścianie nad swoim łóżkiem.
Po wielu latach kolekcjonowania plakatów, zakleił już prawie całą ścianę i ciężko
	jest mu znaleźć na niej trochę wolnego miejsca.

Mirek właśnie kupił sobie trochę nowych plakatów i potrzebuje Twojej pomocy,
	żeby znaleźć dla nich odpowiednie miejsce na ścianie.
Dla każdego nowego plakatu i jego umiejscowienia na ścianie,
	Mirek chciałby wiedzieć jak bardzo ten plakat przykryłby inne plakaty wiszące na ścianie.

\heading{Task}

Masz dane współrzędne wszystkich plakatów, które wiszą na ścianie,
	oraz współrzędne wszystkich plakatów, które jeszcze tam nie wiszą,
	ale Mirek chciałby je powiesić.
Dla każdego nowego plakatu, oblicz sumaryczne pole powierzchni tych części plakatów (wiszących na ścianie),
	które są bezpośrednio przykryte przez ten nowy plakat.
Plakaty mogą na siebie nachodzić, więc jeśli część wspólna dwóch plakatów jest przykrywana,
	to należy tę powierzchnie policzyć tylko raz.

\heading{Input}

W pierwszej linii wejścia znajduje się jedna liczba całkowita $N$ ($1 \le N \le 100\,000$),
	oznaczająca liczbę plakatów, które już wiszą na ścianie.
W następnych $N$ liniach znajdują się opisy tych plakatów.
W $N+2$ linii znajduje się jedna liczba całkowita $M$ ($1 \le M \le 100\,000$),\
	oznaczająca liczbę plakatów, które Mirek chciałby powiesić.
W następnych $M$ liniach znajdują się opisy tych plakatów.

Każdy plakat jest prostokątem, o bokach równoległych do osi układu współrzędnych.
Ten prostokąt jest opisany przez cztery liczby całkowite $x_1$, $y_1$, $x_2$, $y_2$
	($0 \le x_1 < x_2 \le 10^9$, $0 \le y_1 < y_2 \le 10^9$),
	które oznaczają współrzędne lewego dolnego i prawego górnego rogu.

W $12.5\%$ testów zachodzi $n, m \le 10$, oraz współrzędne są nie większe niż $100$.

W $25\%$ testów zachodzi $n, m \le 50$.

W $50\%$ testów zachodzi $n, m \le 1000$.

W $50\%$ testów współrzędne są nie większe niż $30\,000$.

\heading{Output}

Dla każdego nowego plakatu wypisz jedną linię zawierającą jedną liczbę całkowitą
	-- odpowiedź na pytanie Mirka.
Odpowiedzi należy udzielać w tej samej kolejności,
	w jakiej plakaty były podane na wejściu.

\newpage

\heading{Sample}


\sampleIN
2
0 1 3 5
2 3 6 6
2
1 0 5 4
4 2 7 7
\sampleCOMMENT
Ściana Mirka jest przedstawiona na poniższym obrazku.
Wykreskowane prostokąty to nowe plakaty,
	a~wypełnione prostokąty to plakaty, które wiszą na ścianie.
\begin{center}
	\begin{tikzpicture}[scale=0.5]
		\filldraw (0, 0) node[below left] {$(0, 0)$} circle(0.1);
		\filldraw[gray!20] (0, 1) -- (3, 1) -- (3, 5) -- (0, 5) -- cycle;
		\draw (0, 1) -- (3, 1) -- (3, 5) -- (0, 5) -- cycle;
		\filldraw[gray!20] (2, 3) -- (6, 3) -- (6, 6) -- (2, 6) -- cycle;
		\draw (2, 3) -- (6, 3) -- (6, 6) -- (2, 6) -- cycle;
		\draw[dashed] (1, 0) -- (5, 0) -- (5, 4) -- (1, 4) -- cycle;
		\draw[dashed] (4, 2) -- (7, 2) -- (7, 7) -- (4, 7) -- cycle;
	\end{tikzpicture}
\end{center}
\sampleOUT
8
6
\sampleCOMMENT
\sampleEND


\end{document}
