\input sys/inputs.tex

\begin{document}

\bigheading{Triedenie}

% \info{task_name}{infile}{outfile}{points}{timelimit}{memlimit}
% leave this values, if you are not interested
\info{sorting}{stdin}{stdout}{100}{1000 ms}{1 GB}

Milé deťúrence, posadajte si okolo ohníka a vypočujte si príbeh o Bobovi, ktorý nemal čas.

Malý Bob bol strašne netrpezlivý a nerád zabíjal čas vecami, ktoré pokladal za zbytočné. Jedného dňa
sa ocitol na CPSPC\footnote{Obdoba VPCPC z dávnych čias, keď sme sa stretávali ešte bez maďarov.}. A
keďže si myslel, že slovenké preklady sú zdĺhavé a rozvláčne, odmietol ich čítať. Namiesto toho
čítal zadania anglické.

I dostal sa k jednej úlohe. Číta, číta a úloha sa mu zdá ľahká. Však stačí len utriediť čísla na
vstupe od najmenšieho po najväčšie. Pustil sa teda do programovania a za malú chvíľu submitol svoje
riešenie. Aké však bolo jeho prekvapenie, keď zisil, že len na niektorých vstupoch dostal OK.
Myslel si, že má bug vo svojom riešení a tak debugoval a debugoval, všetko neúspešne. Nakoniec si
ešte raz prečítal zadanie a zistil, že si nevšimol malú nevýraznú poznámku, že triediť mal
lexikograficky.

Naprogramoval teda nové riešenie, celý čas však rozmýšľal, aká to bola zhoda náhod, že predsa len
dostal niekoľko OK, keď riešil inú úlohu. Po konci súťaže sa išiel posťažovať svojim kamarátom, ako
nepekne sa okašľal a tí ho namiesto súcitu vysmiali, lebo si čítali slovenské zadanie, kde bolo
slovo \textbf{lexikograficky} poriadne zvýraznené a teda nemali žiaden problém s riešením.

Tým končí náš príbeh. A poučenie? Neverte Bobovi a čítajte slovenské zadania. Nie len, že sú super,
ale častokrát sú viac zrozumiteľné.

\heading{Úloha}

Na vstupe dostanete uzavretý interval celých čísiel $<A,B>$. Nájdite počet podmnožín týchto čísiel
takých, že ich utriedenie lexikograficky a numericky dáva tú istú postupnosť.

\heading{Vstup}

Na prvom riadku sú dve medzerou oddelené čísla $A$ a $B$ ($1 \leq A \leq B \leq 10^{18}$, $B-A \leq
10^5$).

\heading{Výstup}

Vypíšte jedno číslo $M$, kde $M$ je počet podmnožín množiny $\{A, A+1 \dots B\}$, ktoré sa utriedia
rovnako numericky aj lexikograficky. Odpoveď vypíšte modulo $10^9 + 7$.

\heading{Príklad}


\sampleIN
98 101
\sampleOUT
7
\sampleCOMMENT
Hľadané podmnožiny sú: $\emptyset$, $\{98\}$, $\{99\}$, $\{100\}$, $\{101\}$, $\{98, 99\}$, $\{100, 101\}$.
\sampleEND


\end{document}
