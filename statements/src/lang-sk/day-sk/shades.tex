\input sys/inputs.tex

\begin{document}

\bigheading{Tieňe mesta}

% \info{task_name}{infile}{outfile}{points}{timelimit}{memlimit}
% leave this values, if you are not interested
\info{shades}{stdin}{stdout}{100}{1000 ms}{1 GB}

Kde bolo, tam bolo, bolo raz jedno mesto.
\footnote{Sú tu určité pochybnosti o správnosti prekladu.}
Teda\dots{} predpokladáme, že to bolo mesto.
Každopádne, toto mesto zmizlo a nezanechalo nič.
Takmer nič. Iba tieň. Studený, tmavý tieň.
Tieň, ktorý sa nehýbe. Tieň osudu.
Tieň\dots{} OK, dosť bolo rozprávok.
Ideme kódiť.
Potrebujeme zistiť výšky budov v meste.
Tieň však môže byť ľubovoľne natiahnutý.
Našťastie vieme detaily o architektúre civilizácie,
ktorá obývala túto oblasť.

Mesto pozostávalo z budov zoradených do jedného radu.
Rovnako široké, rovnako vzdialené. Len výšky mali rôzne.
A aj tie mali svoje vzory.

Tvojou úlohou bude nájsť všetky výskyty vzorov v tieni
a vypísať ich počet.


\heading{Úloha}

Tieň je postupnosť celých čísel -- naškálovaných výšok.
Vzor je tiež postupnosť celých čísel -- výšok.
Hovoríme že vzor má výskyt v tieni,
ak existuje \textbf{súvislá} podpostupnosť tieňa taká,
že po jej naškálovaní (t.j.\ prenásobení výšok nejakou kladnou reálnou konštantou) dostaneme vzor.
Vzor môže mať v tieni aj viac výskytov, ktoré sa môžu aj prekrývať.


\heading{Vstup}

Prvý riadok vstupu obsahuje celé číslo $n$
-- počet vzorov.

Každý z nasledovných $n$ riadkov popisuje jeden vzor.
Začína sa celým číslom $l_i$ -- dĺžkou vzoru.
Po ňom nasleduje $l_i$ medzerou oddelených kladných celých čísel.

Posledný riadok popisuje tieň.
Začína sa celým číslom $m$ -- dĺžkou tieňa.
Po ňom nasleduje $m$ medzerou oddelených kladných celých čísel.

\bigskip

Platí:\\
$1 \leq m \leq 3\cdot10^5$\\
$1 \leq l_i$\\
$\sum^{n}_{i=1} l_i \leq m$\\
Všetky členy postupností su najviac $10\,000$.

\smallskip

V $40\%$ vstupov dokonca platí $m \leq 1000$.

\heading{Výstup}

Vypíšte jeden riadok s celkovým počtom všetkých výskytov
všetkých vzorov v tieni.

\heading{Príklady}


\sampleIN
4
1 47
2 21 42
2 34 17
3 1 2 1
7 3 6 3 6 12 6 3
\sampleOUT
15
\sampleCOMMENT
Prvý vzor vie byť naškálovaný na akúkoľvek výšku.
Preto sa vyskytuje 7-krát. Druhý s tretím sa
vyskytujú 3-krát a posledný vzor 2-krát.
\sampleEND


\end{document}
