\input sys/inputs.tex

\begin{document}

\bigheading{Hypernice}

% \info{task_name}{infile}{outfile}{points}{timelimit}{memlimit}
% leave this values, if you are not interested
\info{hyperways}{stdin}{stdout}{100}{3000 ms}{1 GB}

Intergalaktická spoločnosť Oods\&co vybuduje 
počas nasledujúcich storočí v našej galaxii
sieť hyperníc.\footnote{hyperpriestorové dialnice, tiež zvané hyhi}
Plány stavby (poradie stavania dialnic) sú už hotové.
Zostáva skontrolovať či sú dodržané
smernice pre bezpečnosť výstavby hyperníc.
A na to je potrebné zistiť zmeny počtov
bezpečných hyperníc počas výstavby.
Vrámci projektu rozvoja zaostalých planét
to budete kódiť vy.

\heading{Úloha}

Hypernica je bezpečná ak neexistujú 2 planéty
medzi ktorými je jedinou cestou.
(Teda ak je súčasťou nejakého hypernicového cyklu.)
Keď pridávame hypernice, počet bezpečných sa môže iba zvýšiť.
Dostanete poradie, v ktorom staviame hypernice.
Po každej stavbe povedzte, koľko hyperníc sa stalo bezpečnými.


\heading{Vstup}

Prvý riadok obsahuje 2 celé čísla $n$ a $m$.
$n$ je počet planét,
$m$ je počet hyperníc.
Planéty sú číslované $1$ \dots $n$.

Každý z nasledujúcich $m$ obsahuje 2
medzerou oddelené celé čísla.
Sú to indexy planét, ktoré spája ďalšia hypernica.
Všetky hypernice sú obojsmerné.
Medzi dvoma planétami môže viesť viacero hyperníc
a hypernica môže viesť aj z plánety na ňu samotnú.

\smallskip

Platí $n \leq 10^6$, $m \leq 2\cdot10^6$.\\
V $40\%$ vstupoch dokonca platí
$n \leq 1000$, $m \leq 2000$.

\heading{Výstup}

Pre každú hypernicu na vstupe vypíšte do samostatného riadku,
koľko hyperníc sa jej vybudovaním stalo bezpečnými (vrátane nej).

\heading{Príklady}


\sampleIN
5 8
1 2
3 3
4 5
2 3
4 5
3 4
4 1
5 2
\sampleOUT
0
1
0
0
2
0
4
1
\sampleEND


\end{document}