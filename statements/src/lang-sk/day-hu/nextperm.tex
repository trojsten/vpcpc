\input sys/inputs.tex

\begin{document}

\bigheading{\texttt{next\_permutation}}

% \info{task_name}{infile}{outfile}{points}{timelimit}{memlimit}
% leave this values, if you are not interested
\info{nextperm}{stdin}{stdout}{100}{100 ms}{32 MiB}

V matematike aj informatike sa permutácie tešia širokej obľube. No nie všetky si
to zaslúžia! Posadajte si, deťúrence, okolo ohníka a vypočujte si príbeh starší
ako ľudstvo samo.

V časoch, keď sa ešte len formulovali axiómy predikátovej logiky, povrch Zeme
brázdili iba elementárne matematické objekty. Neboli vojny, hlad ani choroby,
keďže také elementárne objekty nejedli, nekýchali, ba ani sa im nechcelo bojovať
o to, ktorý z nich je elementárnejší. Žili si v harmónii, netušiac nič o hroznom
osude, ktorý na nich číhal.

Počas nasledujúcich stáročí sa postupne objavovali čoraz zložitejšie objekty,
medzi nimi aj permutácie. Objekty rovnakého druhu sa väčšinou držali pokope a k
cudzincom prechovávali nedôveru až nenávisť. Vytvárali uzavreté grupy, niektoré
dokonca aj gangy. Prestrelky, únosy a binárne operácie sa stali bežnou súčasťou
života. Časom už nebolo jediného matematického objektu, ktorý by zostal
nepoznačený neustálymi bojmi -- dokonca i neutrálne prvky boli vyhnané z úkrytov
v podzemí a donútené chopiť sa zbraní.

Vtedy sa elementárne objekty podujali zastaviť boje a všetkých zmieriť. Na túto
preťažkú úlohu by ste nenašli vhodnejších kandidátov -- elementárne objekty boli
múdre, skúsené, a ešte stále požívali značnú úctu. Spustili ohromnú reklamnú
kampaň, presviedčali zložitejšie objekty na ulici, navštevovali ich doma,
rozšírili slogan ``\emph{Nehľaď na kategóriu, neskončíš v krematóriu!}''.
Zakrátko ich snaha dosiahla úspech a boje ustali.

Vtom sa na scéne objavila permutácia $(3,1,2)$ -- malá, no zaslepená nenávisťou
k všetkému cudziemu. Táto permutácia sa cez vetraciu šachtu prešmykla priamo do
Nukleárneho vojenského centra Symetrickej grupy a odpálila jadrovú hlavicu
namierenú na Spojené kráľovstvo Šťastnej rodiny a Skupiny vyvrheľov. Tento čin
rozpútal svetovú vojnu, ktorá zniesla všetky matematické objekty (elementárne i
zložitejšie) z tohto sveta a poslala ich späť do platónskeho sveta ideí.

Príbehu je koniec. A poučenie? Čítajte radšej anglické zadania, milé deťúrence.
Ušetríte čas.


\heading{Úloha}

Permutácia $p_1, p_2, \dots, p_n$ prirodzených čísel $1, 2, \dots, n$ sa nazýva
vzor-$(3,1,2)$-neobsahujúca, ak neexistujú také tri indexy $i,j,k$, že $1 \leq i
< j < k \leq n$ a $p_j < p_k < p_i$.

Autori súčasnej implementácie \texttt{next\_permutation} v C++ zjavne nepočuli
príbeh o hanebnom čine permutácie $(3,1,2)$, inak by všetky permutácie
obsahujúce vzor $(3,1,2)$ za trest ignorovali. Napravte ich nevedomosť; napíšte
program, ktorý pre danú vzor-$(3,1,2)$-neobsahujúcu permutáciu nájde nasledujúcu
vzor-$(3,1,2)$-neobsahujúcu permutáciu v lexikografickom poradí.


\heading{Vstup}

Prvý riadok vstupu obsahuje jedno celé číslo $n$ ($3 \leq n \leq 10\,000$).
Druhý riadok obsahuje $n$ čísel oddelených medzerami -- nejakú
vzor-$(3,1,2)$-neobsahujúcu permutáciu prirodzených čísel $1, 2, \dots, n$.
Môžete predpokladať, že permutácia na vstupe nie je klesajúcou postupnosťou $n,
n-1, \dots, 2, 1$.

V $40\%$ vstupov navyše platí $n \leq 1\,000$.


\heading{Výstup}

Jediný riadok výstupu má obsahovať vzor-$(3,1,2)$-neobsahujúcu permutáciu,
ktorá nasleduje za permutáciou na vstupe v lexikografickom poradí. Jednotlivé
prvky permutácie oddeľujte jednou medzerou.


\heading{Príklad}

\sampleIN
5
2 4 5 3 1
\sampleOUT
2 5 4 3 1
\sampleCOMMENT

\sampleEND
\bigskip

\end{document}
