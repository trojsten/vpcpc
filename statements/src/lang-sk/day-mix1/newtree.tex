\input sys/inputs.tex

\begin{document}

\bigheading{Nový strom}

% \info{task_name}{infile}{outfile}{points}{timelimit}{memlimit}
% leave this values, if you are not interested
\info{newtree}{stdin}{stdout}{100}{200 ms}{1 GB}

Záhradník zasadil v mestskom parku nový strom, no keďže je ešte malý a slabý,%
\footnote{ten strom}
musí%
\footnote{záhradník}
ho%
\footnote{ten strom}
ochrániť plotom. Ohradená oblasť má mať tvar trojuholníka s vrcholmi v troch
starých stromoch. Nový strom musí byť vnútri chránenej oblasti, no nesmie ležať
na jej hranici. Žiadny zo starých stromov okrem tých vo vrcholoch trojuholníka
nesmie ležať vnútri chránenej oblasti, a to ani na jej hranici.


\heading{Úloha}

Záhradník už vybral jeden starý strom. Nájdite ďalšie dva, ktoré spolu s ním
vytvoria chránenú oblasť.


\heading{Vstup}

Prvý riadok vstupu obsahuje dve celé čísla $N$ a $A$ -- počet starých stromov
($3 \leq N \leq 10^5$) a číslo vybraného starého stromu ($1 \leq A \leq N$).
Staré stromy sú očíslované od $1$ do $N$.

Druhý riadok obsahuje dve celé čísla $x$ a $y$ -- $x$-ovú a $y$-ovú súradnicu
nového stromu. Každý z nasledujúcich $N$ riadkov obsahuje dve celé čísla $x_i$ a
$y_i$ -- súradnice $i$-teho starého stromu. Žiadna zo súradníc v absolútnej
hodnote neprekročí $10^6$.

\bigskip

Navyše máte garantované, že v testovacích vstupoch hodných $40$ bodov bude
platiť $N \leq 5\,000$.


\heading{Výstup}

Na jediný riadok výstupu vypíšte dve celé čísla $B$ a $C$ oddelené jednou
medzerou.

Ak riešenie úlohy neexistuje, potom má platiť $B = C = 0$. V opačnom prípade
majú staré stromy s číslami $A$, $B$ a $C$ tvoriť vrcholy chránenej oblasti v
poradí proti smeru hodinových ručičiek. Ak existuje viacero riešení, vypíšte
vaše najobľúbenejšie.


\heading{Príklad}

\sampleIN
7 1
9 3
3 1
8 7
9 5
11 5
12 4
9 1
13 6
\sampleOUT
6 4
\sampleCOMMENT

\sampleEND
\bigskip

\end{document}
