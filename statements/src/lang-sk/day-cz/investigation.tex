\input sys/inputs.tex

\begin{document}

\bigheading{Vyšetrovanie}

% \info{task_name}{infile}{outfile}{points}{timelimit}{memlimit}
% leave this values, if you are not interested
\info{investigation}{stdin}{stdout}{100}{2500 ms}{1 GB}

V meste Bytelandia sa stala lúpež! Zlodejovi sa podarilo utiecť a skryť sa
niekde v meste. Vy tento zločin vyšetrujete a vašou úlohou je dostať zlodeja za
mreže.

Mesto sa skladá z $n$ domov a $n-1$ ciest, z ktorých každá obojsmerne spája
niektorú dvojicu domov. Mesto má stromovú štruktúru, čiže pre každú dvojicu
domov existuje práve jeden spôsob, ako medzi nimi prejsť (ak teda nechceme
použiť jednu cestu viackrát). Zlodej sa skrýva v niektorom z domov a počas
celého priebehu vyšetrovania nemení svoju polohu.

Aby ste zlodeja našli, môžete si vybrať nejaký dom $h$ a prehľadať ho. Ak sa v
tomto dome skrýval zlodej, rovno ho aj zatknete. V opačnom prípade drastickým
spôsobom vypočujete všetkých obyvateľov domu a dozviete sa hodnotu $c_i$ s
nasledujúcim významom: ``Ak si predstavíme mesto ako strom zakorenený v dome $h$
a označíme synov koreňa $c_1, c_2, \dots, c_m$, potom sa zlodej skrýva v
niektorom z domov v podstrome zakorenenom v $c_i$.'' Domy musíte prehľadávať
dovtedy, kým zlodeja nezatknete.

Také prehľadávanie je vo vašom veku už celkom náročné na kondíciu -- musíte
hrdinsky vykopnúť dvere, oháňať sa zbraňami a popri tom z plného hrdla revať.
Potrebujete si preto vymyslieť optimálnu stratégiu, s ktorou minimalizujete
počet prehľadaných domov v najhoršom možnom prípade.

\heading{Úloha}

Na vstupe dostanete popis mesta. Nájdite stratégiu pre prehľadávanie domov,
ktorá minimalizuje počet prehľadaných domov v najhoršom prípade.

\heading{Vstup}

Prvý riadok obsahuje jedno celé číslo $n$ -- počet domov v meste. Domy sú
očíslované od $0$ od $n-1$.

Druhý riadok obsahuje $n-1$ medzerami oddelených čísel $v_1~v_2~\dots~v_{n-1}$.
Číslo $v_i$ znamená, že existuje cesta medzi domami $i$ a $v_i$. Navyše platí
$v_i < i$.

\smallskip
\noindent
Platí $2 \leq n \leq 10^5$.\\
V $20\%$ vstupov $n \leq 10$.\\
V $40\%$ vstupov $n \leq 20$.\\
V $60\%$ vstupov $n \leq 1000$.

\heading{Výstup}

Na výstup vypíšte práve jedno číslo -- počet domov, ktoré treba prehľadať v
najhoršom možnom prípade, ak budete používať optimálnu stratégiu.

\heading{Príklad}

\sampleIN
5
0 1 1 1
\sampleCOMMENT
Štruktúra mesta vyzerá ako hviezda s domom $1$ uprostred.
\sampleOUT
2
\sampleCOMMENT
Ako prvý prehľadajte dom $1$. Ak tam zlodeja nenájdete, potom sa po drastickom
vypočúvaní dozviete priamo číslo domu, v ktorom sa nachádza.
\sampleEND


\bigskip


\sampleIN
8
0 1 2 1 3 5 6
\sampleOUT
3
\sampleEND


\end{document}
