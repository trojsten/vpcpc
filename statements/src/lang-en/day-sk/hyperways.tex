\input sys/inputs.tex

\begin{document}

\bigheading{Hyperways}

% \info{task_name}{infile}{outfile}{points}{timelimit}{memlimit}
% leave this values, if you are not interested
\info{hyperways}{stdin}{stdout}{100}{3000 ms}{1 GB}

Oods\&co intergalactic company will build a network of hyperways
\footnote{hyperspace highways, also called hyhi}
connecting planets of our galaxy in few next centuries.
They have already prepared an construction plan.
(An order of building hyperways.)
It remains to check whether the plan satisfies
intergalaxy hyhi safety regulations.
To do this, Oods will need to calculate
the number of safe highways
after each step of construction.
Acording to the
Project of local planets integration
you will code it.


\heading{Task}

The hyperway is safe
if it is not the only one connecting
some two planets
(e.g. it lies on some hyperway cycle).
You will be given the order in which hyperways
will be constructed.
After each construction there may
be some hyperways, which have became safe.
Output the number of them.
Note that when you are just adding hyperways,
they can just became safe from unsafe,
not the other way!


\heading{Input}

On the first line of input are two integers $n$ and $m$.
$n$ is the number of planets in the plan,
$m$ is the number of hyperways.
The planets are numbered $1$ \dots $n$.

Each of next $m$ lines consists
of two space separated integers --
the id's of planets next hyperway will join.
All the hyperways are passable in all directions.
There can be a hyperway connecting a planet with itself.
There can be a multiple hyperways between two planets.

\smallskip

It holds $n \leq 10^6$, $m \leq 2\cdot10^6$.\\
In the $40\%$ of testcases also holds
$n \leq 1000$, $m \leq 2000$.

\heading{Output}

Output a line with single integer for each hyperway in the input --
the number of hyperways which will become safe by adding it
(including the hyperway itself).

\heading{Samples}


\sampleIN
5 8
1 2
3 3
4 5
2 3
4 5
3 4
4 1
5 2
\sampleOUT
0
1
0
0
2
0
4
1
\sampleEND


\end{document}