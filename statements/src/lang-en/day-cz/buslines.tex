\input sys/inputs.tex

\begin{document}

\bigheading{Bus lines}

% \info{task_name}{infile}{outfile}{points}{timelimit}{memlimit}
% leave this values, if you are not interested
\info{buslines}{stdin}{stdout}{100}{1000 ms}{1 GB}

In the Czech city called Kocourkov they have a spectacular trafic system.
There are $N$ bus stops and some of them are connected by $N - 1$ bidirectional streets.
These streets intersect only at the bus stops and moreover it is possible to go from every bus stop
to all of the others using a sequence of streets (i.e. the city forms a tree structure).

Kocourkov Public Transportation Company has finally decided to introduce buses.
Their representatives want to make sure that it is possible to travel between all pairs of bus stops.
Because of that there will be a new bus line from each bus stop to each other.
We can easily calculate that there will be $\frac{n \cdot (n - 1)}{2}$ lines in total.
Some of the lines will consist of just two stops, others can have possibly up to $N$ stops.

Everything sounds simple so far, but here comes a technical problem, which workers encountered while building the system.
At each bus stop must be a timetable listing all of the lines which goes though the particular stop.
The workers don't know how large the tables should be.

\heading{Task}

You are given the description of trafic system in the city of Kocourkov.
For every bus stop in the city calculate the number of lines going through it (including lines, which starts or ends here).

\heading{Input}

First line contains an integer $N$, the number of bus stops in the city (stops are numbered from $1$ to $N$).
The following $N - 1$ lines describes streets in the city.
Every such line contains two different integers $1 \le x, y \le N$ meaning that there is a street from bus stop number $x$ to bus stop number $y$.
You can expect that it describes a valid city according to the problem statement.

\smallskip

It holds $1 \leq N \leq 10^6$.\\
In the $20\%$ of testcases $N \leq 20$.\\
In the $40\%$ of testcases $N \leq 1000$.

\heading{Output}

Output $N$ lines. The $i$-th line should contain a signle integer, the number of bus lines going through the bus stop number $i$.

\heading{Samples}


\sampleIN
5
4 5
2 1
3 2
2 5
\sampleOUT
4
7
4
4
7
\sampleEND


\bigskip


\sampleIN
6
1 2
2 3
3 4
4 5
5 6
\sampleOUT
5
9
11
11
9
5
\sampleEND


\end{document}
