\input sys/inputs.tex

\begin{document}

\bigheading{Hypernice}

% \info{task_name}{infile}{outfile}{points}{timelimit}{memlimit}
% leave this values, if you are not interested
\info{hyperways}{stdin}{stdout}{100}{3000 ms}{1 GB}

Intergalaktická společnost Oods\&co vybuduje
během následujícího století sít hypernic,\footnote{hyperprostorové dálnice, takzvané hyhi}
které budou spojovat planety v naší galaxii.
Již je také hotový plán stavby, t.j. pořadí stavby hypernic.
Každá hypernice bude obousměrný tunel spojující dvě (ne nutně různé) planety.

\heading{Úloha}

Hypernice je \emph{bezpečná}, pokud není jedinou hypernicí spojující (přímo nebo nepřímo) nějakou dvojici planet.
Jinými slovy, hypernice $H$ je \emph{riziková} pokud existuje dvojice planet $A$ a $B$ taková, že
pokud cestujeme z $A$ do $B$, tak musíme použít hypernici $H$.
(Povšimněte si, že bezpečné hypernice jsou ty, které jsou součástí nějakého cyklu.)

Na vstupu dostanete pořadí, ve kterém budou hypernice postaveny.
Po každé stavbě se mohou nějaké hypernice stát bezpečnými.
(Včetně, možná ale ne nutně, právě vybudované hypernice.)
Spočítejte je.

Povšimněte si, že hypernice, které již bezpečné jsou, tak bezpečnými zůstanou až do konce stavby.

\heading{Vstup}

První řádek obsahuje 2 celá čísla $n$, $m$.
$n$ je počet planet,
$m$ je počet hypernic.
Planety jsou číslované $1 \dots n$.

Každý z následujících $m$ řádků obsahuje 2
mezerou oddělená celá čísla --
indexy planet, které spojuje další hypernice.
Všechny hypernice jsou obousměrné.
Mezi dvěma planetami může vézt více hypernic
a hypernice může vézt i z planety do ní samotné.

\bigskip

\noindent
Platí $n \leq 10^6$, $m \leq 2\cdot10^6$.\\
Ve $40\%$ vstupů dokonce platí
$n \leq 1000$, $m \leq 2000$.

\heading{Výstup}

Pro každou hypernici na vstupu vypište na samostatném řádku
kolik hypernic se jejím vybudováním stalo bezpečnými (včetně ní).

\heading{Příklady}


\sampleIN
5 8
1 2
3 3
4 5
2 3
4 5
3 4
4 1
5 2
\sampleOUT
0
1
0
0
2
0
4
1
\sampleEND


\end{document}
