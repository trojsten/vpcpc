\input sys/inputs.tex

\begin{document}

\bigheading{Následující permutace}

% \info{task_name}{infile}{outfile}{points}{timelimit}{memlimit}
% leave this values, if you are not interested
\info{nextperm}{stdin}{stdout}{100}{100 ms}{32 MiB}

Permutace jsou častým objektem zájmu v matematice a informatice\footnote{Vzpomeňte na úlohu RubiKubismus z pondělka.}.
Permutace s určitými vlastnostmi jsou obzvlášť zajímavé.
Permutace $p_1,p_2, \ldots, p_n$ čísel $1, \ldots n$ se nazývá 3-1-2 uhýbající pokud neexistuje trojice indexů $1\leq i<j<k \leq n$ taková, že $p_i>p_j$,  $p_i>p_k$ a $p_j<p_k$.

\heading{Úloha}

Napište program, který pro danou 3-1-2 uhýbající permutaci spočte následující 3-1-2 uhýbající permutaci podle lexikografického uspořádání.

\heading{Vstup}

První řádek vstupu obsahuje jedno celé číslo $N$ ($3 \leq N \leq 10\,000$).

Druhý řádek vstupu obsahuje $N$ mezerou oddělených čísel, 3-1-2 uhýbajíci permutaci čísel $1, \ldots n$.
Múžete předpokládat, že vstupem není klesajíci posloupnost $n, n-1, \ldots 1$.

\smallskip

Ve $40\%$ případů platí $n \leq 1\,000$.

\heading{Výstup}

Jediný řádek výstupu obsahuje výslednou 3-1-2 uhýbající permutaci.

\heading{Příklady}

\sampleIN
5
2 4 5 3 1

\sampleOUT
2 5 4 3 1
\sampleCOMMENT

\sampleEND
\bigskip

\end{document}
