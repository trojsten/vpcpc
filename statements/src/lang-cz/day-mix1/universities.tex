\input sys/inputs.tex

\begin{document}

\bigheading{Univerzity}

% \info{task_name}{infile}{outfile}{points}{timelimit}{memlimit}
% leave this values, if you are not interested
\info{universities}{stdin}{stdout}{100}{1000 ms}{1 GB}

Právě jsi odmaturoval a přemýšlíš, na kterou univerzitu se zapsat.
V krajině Bytelandia je $N$ univerzit.
Na každé univerzitě se vyučuje buď bílá nebo černá magie.
Mezi univerzitami vede $N-1$ obousměrných cest, každá cesta medzi dvěma různými univerzitami,
tak aby bylo možné projít mezi každou dvojicí univerzit.

V rámci výběru, na kterou univerzitu se zapsat, plánuješ některé z nich navštívit.
Zvolíš si dvě různe univerzity, jednu startovní a jednu cílovou, a navštíviš
všechny unverzity na cestě\footnote{taková cesta existuje vždy právě jedna} mezi
startovní a cílovou univerzitou, včetně nich.
Aby byla rovnováha mezi silami zachováná, počet navštívených univerzit, které
vyučují bílou magii musí být stejný jako počet navštívených univerzit, které
vyučují černou magii.

Pro každou univerzitu je také znám její rating. Svou cestu po univerzitách chceš
naplánovat tak, aby součet ratingů univerzit, které navštívíš byl co největší.

\heading{Úloha}

Na vstupu dostaneš popis univerzit a cest mezi nimi.
Nalezni optimálni cestu po univerzitách, ve které navštívíš stejný počet
univerzit černé a bílé magie a aby součet ratingů navštívených univerzit
byl co největší.

\heading{Vstup}

Vstup obsahuje čtyři řádky.

První řádek obsahuje jedno číslo $N$ ($2 \leq N \leq 10^5$), počet univerzit (univerzity jsou číslované od $1$ do $N$).

Druhý řádek obsahuje řetězec délky $N$, obsahující znaky ``B'' a ``W''.
Pokud je i-tý znak ``B'', tak i-tá univerzita vyučuje černou magii.
Pokud je i-tý znak ``W'', tak i-tá univerzita vyučuje bílou magii.
Pro každý typ magie, existuje alespoň jedna univerzita, která ji vyučuje.

Třetí řádek obsahuje $N$ mezerou oddělených čísel $h_1 h_2 \ldots h_N$ ($-10^5 \leq h_i \leq 10^5$).
Číslo $h_i$ znamená rating i-té univerzity.

Čtvrtý řádek obsahuje $N-1$ mezerou oddělených čísel $v_1 v_2 \ldots v_{N-1}$.
Číslo $v_i$ znamená, že existuje cesta, která vede mezi univerzitami číslo $v_i$ a $(i+1)$ ($1 \leq v_i \leq i$).


\heading{Výstup}

Na výstup vypište jedno číslo, součet ratingů navštívených univerzit v optimální cestě.

\heading{Příklady}


\sampleIN
6
BWBBBW
6 0 3 -2 100 5
1 2 2 4 4
\sampleOUT
9
\sampleCOMMENT
Optimální je navštívit univerzity 1,2,4,6.
\sampleEND

\bigskip

\sampleIN
3
WBW
1 -10 5
1 2
\sampleOUT
-5
\sampleCOMMENT
Protože musiš navštívit alespoň nějakou univerzitu, může odpověď vyjít i záporná.
\sampleEND

\end{document}
