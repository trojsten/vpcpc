\input sys/inputs.tex

\begin{document}

\bigheading{Krájení dortu}

% \info{task_name}{infile}{outfile}{points}{timelimit}{memlimit}
% leave this values, if you are not interested
\info{cutting}{text files}{text files}{100}{-}{-}

Mimino slaví narozeniny!
Přestože je to skutečně skvělý programátor, tak mu je teprve $N$ let.
Kamilin kompilátor (pamatujete ze včera?) mu upekl velký\footnote{Dort je tak velký, že si jej můžeme představit jako nekonečnou rovinu.}
čokoládový dort a dal na něj $N$ svíček.
Mimino a jeho kamarádi už mají fakt hlad a chtěli by dort nakrájet.
Dort budou krájet rovnými řezy tak, aby na každém kusu dortu zbyla maximálně jedna svíčka.

Kolik nejméně řezů je potřeba udělat, aby byly splněny výše uvedené podmínky?

\heading{Úloha}

Je daná rovina a body na ní. Nalezněte množinu přímek rozdělujících rovinu na souvislé oblasti tak, aby každá oblast obsahovala maximálně $1$ bod.
Počet přímek má být co nejmenší.

Souřadnice bodů jsou všechny celočíselné.
Přímky jsou zadány jako dvojice bodů (s celočíselnými souřadnicemi).
Každá přímka jde směrem od prvního bodu k druhému; to je důležité v případě, že přímka protíná nějaký bod. V takovém případě se bere, že bod skončil na pravé straně přímky.

Prímky mohou být pouze horizontální, vertikální nebo diagonální.

\heading{Vstup}

První řádek obsahuje jedno celé číslo $N$, počet bodů v rovině.
Dalších $N$ řádků popisuje body.

$(i+1)$-ní řádek obsahuje mezerou oddělené souřadnice $i$-tého bodu, $x$ a $y$.

\heading{Výstup}

První řádek obsahuje číslo $L$, počet přímek.
Dalších $L$ řádků popisuje přímky.

$(i+1)$-ní řádek obsahuje čtyři mezerou oddělená čísla $X_1$, $Y_1$, $X_2$, $Y_2$. Body ($X_1$, $Y_1$) a ($X_2$, $Y_2$) musí být různé. $i$-tá přímka prochází oběma body. Alespoň jedna z následujících podmínek musí být splněna:

\begin{itemize}
  \item $X_1 = X_2$ (vertikální přímka)
  \item $Y_1 = Y_2$ (horizontální přímka)
  \item $X_1 - X_2 = Y_1 - Y_2$
  \item $X_1 - X_2 = Y_2 - Y_1$
\end{itemize}

Vaše řešení může obsahovat nejvíce $10\,000$ přímek a absolutní hodnota všech souřadnic musí být menší nebo rovna $1\,000\,000$.

\heading{Bodování}

Pro tuto úlohu máte všechny vstupy k dispozici. Ze submitovacího systému si můžete stáhnout všech $10$ testovacích vstupů. Submitujete pouze výstup v textovém souboru.

Pokud vaše řešení nedodrží výše popsaný formát nebo nerozdělí body do samostatných oblastí, dostanete skóre $0$.

V opačnem případě je vaše skóre pro daný vstup rovné
$ 10 \cdot \left(1 - \sqrt{1 - L_{min} / L}\right)$,
kde $L$ je počet přímek ve vašem řešení a $L_{min}$ je počet přímek v nejelepším řešení submitnutém během soutěže.
Tohle skóre se vyhodnotí až po skončení soutěže.

\heading{Příklad}

\sampleIN
4
3 1
4 5
6 6
8 4
\sampleOUT
2
3 2 8 7
2 8 8 2
\sampleEND

\end{document}
