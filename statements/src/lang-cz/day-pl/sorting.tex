\input sys/inputs.tex

\begin{document}

\bigheading{Sorting}

% \info{task_name}{infile}{outfile}{points}{timelimit}{memlimit}
% leave this values, if you are not interested
\info{sorting}{stdin}{stdout}{100}{5000 ms}{64 MB}

Tvůj kamarád Mirek našel na disku soubor obsahující celá čísla.
Rozhodl se v tom udělat trochu pořádek a setřídit je v rostoucím pořadí.
Jelikož je Mirek poměrně zdantný ajťák, vyzkoušel okamžitě využít konzolový nástroj,
který by problém okamžitě rozlousknul. Není těžké uhodnout jméno toho nástroje,
ale bohužel to nefungovalo tak, jak Mirek očekával
-- poté, co setřídil soubor setřídil, tak zjistil, že program pracuje
dokonce i s čísly jako s řetězci a řadí je tedy lexikograficky.
Mirek se obával, že něco takového by se mohlo stát, ale stejně byl překvapen tím,
že čísla byla beztak uspořádána v rostoucím pořadí podle číselné hodnoty.

Teď má se Mirek diví, jaké měl štěstí a jak se vůbec může stát,
že nějaká čísla mají stejné lexikografické pořadí jako to číselné.
Pomocte mu porozumnět této kuriozitě.

\heading{Úloha}

Je dán rozsah čísel $[A, B]$, spočítejte kolik podmbožin celých čísel z tohoto rozsahu
má stejné lexikografické a číselné pořadí.

\heading{Vstup}

Na jediném řádku vstupu se nacházejí dvě celá čísla $A$ a $B$
($1 \le A \le B \le 10^{18}$, $B - A \le 10^5$).

\heading{Výstup}

Vypište jeden řádek s počtem podmnožin množiny $\{A, A+1, \ldots, B\}$,
splňujících popsanou podmínku.
Neboť odpověd může být opravdu velká, vypište její zbytek po dělení číslem $10^9 + 7$.

\heading{Příklad}

\sampleIN
98 101
\sampleOUT
7
\sampleCOMMENT
Hledané podmnožiny jsou: $\emptyset$, $\{98\}$, $\{99\}$, $\{100\}$, $\{101\}$, $\{98, 99\}$, $\{100, 101\}$.
\sampleEND


\end{document}
