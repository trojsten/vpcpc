\input sys/inputs.tex

\begin{document}

\bigheading{Buszjáratok}

% \info{task_name}{infile}{outfile}{points}{timelimit}{memlimit}
% leave this values, if you are not interested
\info{buslines}{stdin}{stdout}{100}{1000 ms}{1 GB}

Kocourkov város közlekedési hálózata nagyon speciális. $N$ buszmegálló van, és $N-1$ közvetlen kétirányú utca, amely két buszmegállót köt össze. Bármely két buszmegálló között pontosan egy útvonal van.

Minden reggel minden egyes buszmegállóból $N-1$ busz indul az összes többi buszmegállóba. Az $A$-ból $B$-be és a $B$-ből $A$-ba közlekedő buszjáratok különbözőnek tekintendők. Minden busz mindegyik megállóban megáll a kezdő- és végállomást összekötő útvonalon (beleértve a kezdő- és végállomást is).

Minden megállóba ki kell függeszteni, hogy az adott megállóban hány járat áll meg.

\heading{Feladat}

Írj olyan programot, amely minden megállóra kiszámítja, hogy ott hány buszjárat áll meg!

\heading{Bemenet}

A bemenet első sora egy egész számot tartalmaz, a buszmegállók $N$ számát. A buszmegállókat az $1,\ldots,N$ számokkal azonosítjuk.\\
A következő $N-1$ sor adja meg a város utcáit. Mindegyik sor két különböző egész számot tartalmaz: $x \,\, y$  ($1 \le x, y \le N$), ami azt jelenti, hogy az $x$ és $y$ buszmegállók között van közvetlen utca.

\heading{Korlátok:}

\begin{itemize}
  \item $1 \leq N \leq 10^6$.
  \item A tesztesetek $20\%$-ában  $N \leq 100$.
  \item A tesztesetek $40\%$-ában $N \leq 1000$.
\end{itemize}

\heading{Kimenet}
A kimenet pontosan $N$ sort tartalmazzon. Az $i$. sor azon buszjáratok számát tartalmazza, amelyek megállnak az $i$. buszmegállóban!

\heading{Példa bemenet és kimenet}

\sampleIN
6
1 2
2 3
3 4
4 5
5 6
\sampleOUT
10
18
22
22
18
10
\sampleEND

\bigskip

\sampleIN
5
4 5
2 1
3 2
2 5
\sampleOUT
8
18
8
8
14
\sampleEND

\end{document}
